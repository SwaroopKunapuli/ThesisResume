\section*{Résumé}
\label{sec:summary}
Since their conception Deep Eutectic Solvents (DES) have been a constant subject of interest for researchers working on tailor-made solvents for various applications including catalysis, biomass conversion and drug delivery.
The presence of diverse type of constituent molecules that make up these solvents and a spectrum of non-covalent interactions between these constituents help in designing solvents for particular applications by fine-tuning the functional groups of the molecules.
The experiments developed to understand the nature of DES are generally accompanied by a computational study to aid in interpretation of the experimental results.
In this thesis, we set out to understand the structure and formation of two types of DES using classical molecular dynamics, quantum mechanical calculations and fragmentation simulations of non-covalent clusters.
The first part of this work involved parameterization of non-bonding parameters of GAFF v2.11 forcefield for molecular dynamics simulations of Urea : Choline Chloride (2:1 molar ratio) (named Reline) to accurately represent the thermophysical properties (density, diffusion coefficients and viscosity) and the structure of the solvent relative to the experimental results. 
The quantum mechanical calculations involving DFT and semi-empirical methods were used in the second part of this work for thermochemical calculations of non-covalent clusters of Reline and Evernic/Usinic acid to corrobarate the stability results observed in the soft-ionisation mass spectrometric experiments.
Finally a major part of this thesis involves adopting and adapting the algorithm for molecular fragmentation simulations (M$_{3}$C) to make the computational study of fragmentation of non-covalent in mass spectrometry computationally tractable. Despite the initial computational bottleneck of the fragmentation simulations, with a combination of semi-empirical methods and density functional approximations methods we are able to finalize a workflow that can be employed in calculating the breaking curves of the chemical systems at the scale of supramolecular assemblies in mass spectrometry on a high-performance computing (HPC) cluster.
