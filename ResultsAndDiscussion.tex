\subsection*{Results and Discussion}
\subsubsection*{Parametrization of the GAFF v2.11 forcefield for Reline}
\subsubsection*{Thermochemical calculations of non-covalent clusters}
Along with the normal mass-spectra, the other form experimental information of non-covalent clusters available for intrepretation of the stability of certain clusters with theoretical calculations is the variation of these cluster-ion intensities at a range of orifice 1 (OR1) voltages.
 \\ Two forms of comparison can be made based on the available experimental breaking curves: (i) cross-species comparison and (ii) intra-sample comparison. (i) The cross-species comparison involves comparing the experimental stability of urea-containing clusters and corresponding thiourea-containing clusters. (ii) The intra-sample comparison between different cluster species observed in the same experiment. This is a straight-forward comparison between the fraction of ion intensities of each species at a range of OR1 voltages.
\\
 The preliminary approach to understand the stability of certain clusters against the in-source collision-induced dissociation was calculating the interaction energies of these clusters using quantum chemical methods. \\

\subsubsection*{Fragmentation simulations of non-covalent clusters}