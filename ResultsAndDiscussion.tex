\subsection*{Results and Discussion}
\subsubsection*{Forcefield parametrization for Reline}
Since it was demonstrated that the incorporation of empirical charge-scaling can be avoided in molecular dynamic simulations by parametrization of Lennard-Jones(LJ) parameters of GAFF v2.11 forcefield in certain deep eutectic systems, this project started with adopting that approach for Reline system.
The additional parametrization required for this specific system is the LJ parameters of Hydrogen and Nitrogen atoms of urea that participate in interactions extensively. After surveying the parameter hyperspace and incorporating the non-linear structure of urea in Reline, we were able to simulate a system with resonable agreement with experimental density (3.5 $\%$ relative error). The structure of the simulated system determined by the radial distribution functions (RDFs) also show resonable agreement with the experimental RDFs.
\subsubsection*{Thermochemical calculations of non-covalent clusters}
Along with the normal mass-spectra, the other form experimental information of non-covalent clusters available for intrepretation of the stability of certain clusters with theoretical calculations is the variation of these cluster-ion intensities at a range of orifice 1 (OR1) voltages.
 \\ Two forms of comparison can be made based on the available experimental breaking curves: (i) cross-species comparison and (ii) intra-sample comparison. (i) The cross-species comparison involves comparing the experimental stability of urea-containing clusters and corresponding thiourea-containing clusters. (ii) The intra-sample comparison between different cluster species observed in the same experiment. This is a straight-forward comparison between the fraction of ion intensities of each species at a range of OR1 voltages.
\\
 The preliminary approach to understand the stability of certain clusters against the in-source collision-induced dissociation was calculating the interaction energies of these clusters using quantum chemical methods. \\

\subsubsection*{Fragmentation simulations of non-covalent clusters}
added from the workstation